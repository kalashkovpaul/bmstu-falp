\chapter*{Введение}
\addcontentsline{toc}{chapter}{Введение}

%\section*{Что такое ClosureScript}

\textbf{Что такое ClosureScript?}

ClosureScript \cite{closurescript} --- компилятор для Closure \cite{closure}, выдающий в результате код на JavaScript.

\textbf{Что такое Closure?}

Closure --- диалект языка LISP, являющийся динамическим компилируемым языком программирования, поддерживающий доступ к фреймворкам, написанным на Java. Из-за своего родства с LISP поддерживает функциональное программирование и использование макросов.

\textbf{Что нужно для  того, чтобы начать писать на Closure?}

Во-первых, среда разработки или текстовый редактор для Closure ---  подходящих несколько, например Emacs, 	Intellij IDEA, VS Code. В рамках данного мануала будет рассмотрено использование текстового редактора VS Code для работы с Closure.

Во-вторых, сам Closure --- он доступен для установки под MacOS, Linux и  Windows.

В-третьих, 
