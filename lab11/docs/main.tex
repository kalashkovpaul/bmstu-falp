%\chapter{Лис}
\section*{Задание}

Используя хвостовую рекурсию, разработать (комментируя назначение аргументов) эффективную программу, позволяющую:
\begin{enumerate}
	\item Найти длину списка (по верхнему уровню);
	\item Найти сумму элементов числового списка;
	\item Найти сумму элементов числового списка, стоящих на нечетных позициях исходного
списка (нумерация от 0);
	\item Сформировать список из элементов числового списка, больших заданного значения;
	\item Удалить заданный элемент из списка (один или все вхождения).
	\item Объединить два списка.
 \end{enumerate}

Убедиться в правильности результатов.

Для одного из вариантов ВОПРОСА уметь составить таблицу, отражающую конкретный
порядок работы системы (№ шага, состояние резольвенты, для каких термов запускаетя алгоритм унификации, дальнейшие действия: прямой ход или откат).

\clearpage

\section*{Решение}

\begin{lstinputlisting}[
	caption={Листинг программы},
	label={lst:t1},
	linerange={1-64},
	]{../src/main.pro}
\end{lstinputlisting}

